\begin{table}[htpb!] \centering
\caption{Summary Statistics} \label{TEENtab:SumStats}
\begin{tabular} {@{\extracolsep{5pt}}lp{1cm}ccc}\\ [-1.8ex]
\hline\hline\\ [-1.8ex] &&No Pill&Pill&Total \\
&&Available&Available& \\ \midrule 
\textsc{Municipality Characteristics} &&&& \\
&&&& \\
Poverty &&16.4&17.0&16.5\\
&&(7.48)&(7.71)&(7.52) \\
Conservative &&0.285&0.291&0.286\\
&&(0.451)&(0.454)&(0.452) \\
Education Spending &&4,762&5,234&4,838\\
&&(5,479)&(5,482)&(5,482) \\
Health Spending &&1,842&2,333&1,921\\
&&(2,595)&(2,830)&(2,640) \\
Out of School &&4.07&3.99&4.06\\
&&(3.17)&(3.10)&(3.15) \\
Female Mayor &&0.119&0.135&0.121\\
&&(0.323)&(0.342)&(0.326) \\
Female Poverty &&60.4&60.7&60.5\\
&&(10.61)&( 9.64)&(10.5) \\
Pill Distance &&5.11&0.00&4.29\\
&&(16.2)&( 0.0)&(15) \\
&&&& \\
\textsc{Individual Characteristics} &&&& \\
&&&& \\
Live Births &&0.054&0.054&0.054\\
&&(0.226)&(0.226)&(0.226) \\
Fetal Deaths &&0.0562&0.0457&0.0545\\
&&(0.27)&(0.24)&(0.266) \\
Birthweight &&3322.7&3334.3&3324.7\\
&&     (540.0)&     (542.3)&     (540.4)\\
Maternal education  &&       11.92&       12.03&       11.94\\
&&     (2.967)&     (2.894)&     (2.955)\\
Percent working     &&       0.295&       0.395&       0.312\\
&&     (0.456)&     (0.489)&     (0.463)\\
Married     &&       0.340&       0.309&       0.335\\
&&     (0.474)&     (0.462)&     (0.472)\\
Age at Birth      &&       27.05&       27.15&       27.07\\
&&     (6.777)&     (6.790)&     (6.779)\\ \midrule
N Comunas && 346 &224& 346 \\
N Fetal Deaths &&9,846&1,541&11,387\\
N Births &&1,188,579&202,986&1,391,565\\
\hline \hline \\[-1.8ex]
\multicolumn{5}{p{13.3cm}}{\begin{footnotesize}\textsc{Notes:}
Group means are presented with standard deviations below in
parentheses.  Poverty refers to the \% of the municipality
below the poverty line, conservative is a binary variable
indicating if the mayor comes from a politically conservative
party
health and education spending are measured in thousands
of Chilean
pesos, and pill distance measures the distance (in km) to the
nearest municipality which reports prescribing emergency
contraceptives.  Pregnancies are reported as \% of all women
giving live birth, while fetal deaths are reported per live
birth.  All summary statistics are for the period 2006-2011.
\end{footnotesize}} \normalsize\end{tabular}\end{table}
