\begin{table}[!htbp] \centering
\caption{The Effect of the EC Pill on Birth Rates}
\label{TEENtab:aggregateASFR}
\begin{tabular}{@{\extracolsep{5pt}}lcccc}
\\[-1.8ex]\hline \hline \\[-1.8ex] 
& Birth& Birth& Birth& Birth\\
& Rate & Rate & Rate & Rate \\
&(1)&(2)&(3)&(4) \\ \hline
\multicolumn{5}{l}{\textbf{
\noindent Panel A: All Women}} \\
Emergency Contraceptive Pill     &$-$1.193$^{***}$&$-$2.247$^{***}$&$-$1.562$^{***}$&$-$1.721$^{***}$\\
            &[0.425]&[0.489]&[0.524]&[0.620]\\
 & & & & \\
Observations&2,210&2,210&2,210&2,210\\
Mean Birth Rate&53.87&53.87&53.87&53.87\\
 & & & & \\
\multicolumn{5}{l}{\noindent \textbf{
Panel B: 15-19 year olds}} \\
Emergency Contraceptive Pill&$-$3.811$^{***}$&$-$4.546$^{***}$&$-$2.795$^{**}$&$-$3.536$^{**}$\\
            &[0.714]&[1.179]&[1.218]&[1.538]\\
 & & & & \\
Observations&2,205&2,205&2,205&2,205\\
Mean Birth Rate&52.00&52.00&52.00&52.00\\
 & & & & \\
\multicolumn{5}{l}{\noindent \textbf{
Panel C: 20-34 year olds}} \\
Emergency Contraceptive Pill&$-$2.491$^{***}$&$-$3.277$^{***}$&$-$2.273$^{**}$&$-$2.452$^{**}$\\
            &[0.762]&[0.891]&[0.938]&[1.116]\\
 & & & & \\
Observations&2,210&2,210&2,210&2,210\\
Mean Birth Rate&85.49&85.49&85.49&85.49\\
 & & & & \\
\multicolumn{5}{l}{\noindent \textbf{
Panel B: 35-49 year olds}} \\
Emergency Contraceptive Pill&0.240&$-$0.586&$-$0.401&$-$0.106\\
            &[0.274]&[0.411]&[0.455]&[0.600]\\
 & & & & \\
Observations&2,210&2,210&2,210&2,210\\
Mean Birth Rate&21.40&21.40&21.40&21.40\\
\hline \\[-1.8ex] 
{\small Year \& Comuna FEs}             &Y&Y&Y&Y \\
{\small Municipal-Specific Linear Trends}& &Y&Y&Y \\
{\small Time Varying Controls}           & & &Y&Y \\
{\small Spillovers}                      & & & &Y \\
\hline \hline \\[-1.8ex]
\multicolumn{5}{p{12.8cm}}{\begin{footnotesize}
\textsc{Notes:} Each panel presents population-weighted
 difference-in-difference results for a regression of 
age-specific fertility rates (ASFR) on the EC reform for
 the age group in
 each municipality.  ASFR is defined as the number of   
births per 1,000 women.  In the case of all women, this 
is called the General Fertility Rate (GFR). All models  
are estimated by OLS, and each municipality is          
weighted by the population of women. Time varying       
controls included in the regression consist of party    
dummies for the mayor in power, the mayor's gender, the
 vote margin of the mayor, the percent of girls out of  
highschool, education spending spending by both the     
municipality and the Ministry of Education, total       
health spending and health spending on staff and        
training, the percent of female heads of households     
living below the poverty line, the percent of female    
workers in professional positions in the Municipality,  
and condom availability (measured at the level of the   
region). Standard errors are clustered at the level of  
the municipality.
$^{*}$p$<$0.1; $^{**}$p$<$0.05; $^{***}$p$<$0.01\end{footnotesize}}
\normalsize\end{tabular}\end{table}
